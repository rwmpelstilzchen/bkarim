\documentclass[12pt]{article}
\usepackage[a4paper, left=0.2cm, right=0.2cm, top=1cm, bottom=1cm]{geometry}
\pagestyle{empty}

\usepackage{setspace}
\setlength\parskip{0pt}
\setlength\parindent{0pt}

\usepackage{pbox}
\usepackage{multirow}

\newcommand{\middlebox}[1]{$\vcenter{\hbox{#1}}$}

\usepackage{graphicx, xcolor, wallpaper}
\usepackage{tikz}
\usetikzlibrary{arrows, decorations.markings, positioning}
\usetikzlibrary{shapes.callouts}
\definecolor{cevaA}{HTML}{8B6639}
\definecolor{cevaB}{HTML}{029044}
\definecolor{daf}{HTML}{C0DA91}

%\pagecolor{cevaA}
\pagecolor{daf}
\color{cevaA}

%\usepackage{eso-pic}
%\newcommand\BackgroundPic{
%\put(0,0){
%\parbox[b][\paperheight]{\paperwidth}{%
%\vfill
%\centering
%\includegraphics[width=\paperwidth,height=\paperheight]{nijar.jpg}%
%\vfill
%}}}

\usepackage{fontspec}
\usepackage[RTLdocument]{bidi}
\setmainfont[Script=Hebrew, AutoFakeSlant=-0.15]
	{Rutz_OE}
	\setmonofont[Scale=0.8]{PragmataPro}
\newcommand{\hl}[1]{\textbf{\addfontfeature{Color=cevaB}#1}}
\renewcommand{\L}[1]{\LR{\fontspec{Vesper Pro}#1}}
\newcommand{\entypo}[1]{{\fontspec[Color=cevaB]{Entypo}#1}}
\newcommand{\symbolglyph}[1]{{\fontspec[Color=cevaB]{Symbola}#1}}
\usepackage{url}

% for double arrows a la chef
% adapt line thickness and line width, if needed
%\tikzstyle{vecArrow} = [thick, decoration={markings,mark=at position 1 with {\arrow[semithick]{open triangle 60}}},
%   double distance=1.4pt, shorten >= 5.5pt,
%   preaction = {decorate},
%   postaction = {draw,line width=1.4pt, white,shorten >= 4.5pt}]
\tikzstyle{vecArrow} = [
	line width = 8pt,
	decoration={markings, mark = at position 1 with {\arrow[line width=5pt, color = cevaA!50!cevaB]{triangle 60}}},
	shorten >= 5.5pt,
	color = cevaA!50!cevaB,
	preaction = {decorate}
]



\begin{document}
\TileWallPaper{20pt}{20pt}{nekuda.eps} % By Mauricio Duque <http://www.snap2objects.com/2009/05/free-clothing-vector-silhouettes/>
%\AddToShipoutPicture{\BackgroundPic}

\begin{center}
\color{cevaA}
\fontsize{30}{36}\selectfont

\begin{tikzpicture}[thick]
	\node (a) {\parbox{15cm}{\centering\setRL\color{cevaA} רוצים להפגש עם משפחות\\
	נוספות עם ילדים \hl{בשעות הבוקר}?}};
	\node [below = 0.2cm of a] (b) {\parbox{15cm}{\centering\setRL\color{cevaA} בחינוך ביתי ומחפשים \hl{חֶבְרָה}?}};
	\node [inner sep=0,minimum size=0,right = of a] (x) {}; % invisible node
	\node [inner sep=0,minimum size=0,left = of b] (y) {}; % invisible node
	%\draw [vecArrow] (a) to (x);
	\draw [vecArrow] (a) |- (x) |- (b);
	\draw [vecArrow] (b) to (y) |- (a);
	%\draw [->] (y) |- (a);
\end{tikzpicture}

\vfill

\vspace{-1cm}
{\fontsize{36}{43}\selectfont
\textbf{\textcolor{cevaA}{אתם מוזמנים להצטרף אל}}\\
{\fontspec[Script=Hebrew]{HPLJPM+AlenbiSerif}\fontsize{75}{90}\selectfont %\textcolor{cevaA}{ל}
\textcolor{cevaB}{מפגשים לילדים}}\\
\vspace{-0.0em} \textbf{המתקיימים דרך קבע \hl{בבקרים}} (מ־10:00)
}

\vfill

בימי \hl{שני} בבית יהודית ובגן שכטר הסמוך\\
בימי \hl{חמישי} בגינה הקהילתית במוזיאון הטבע\\
בימי \hl{ראשון} במשחקיה שבבי״ס תל״י (ק.~היובל)

\vfill

המפגשים \hl{רב־גילאיים} ופתוחים \hl{לכולם}

\vfill

פרטים בדף שלנו ב\hl{אינטרנט}:

\includegraphics[width=0.15\textwidth]{qr-bkarim.png}\\
\vspace{-0.5cm}\L{\texttt{\middlebox{\addfontfeature{Color=cevaA}www.}tiny.cc/bkarim}}

\vfill

\fontsize{20}{24}\selectfont
\begin{LTR}
\begin{tabular}{ccc}
\multirow{3}{*}{\addfontfeature{Color=cevaB, Scale=3}\}} & \L{Kunvenoj por hejminstruataj / liberedukataj infanoj} & \multirow{3}{*}{\addfontfeature{Color=cevaB, Scale=3}\{}\\
	~ & \L{Get-togethers for home-/unschooled children} & ~\\
	~ & \RL{\fontspec[Script=Arabic, Language=Arabic, Scale=1.4]{Scheherazade}\textbf{لقاءات للأهل والاولاد في الصباح}} & ~
\end{tabular}
\end{LTR}

\end{center}
\end{document}
