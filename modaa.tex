\documentclass[12pt]{article}
\usepackage[a4paper, left=0.2cm, right=0.2cm, top=1cm, bottom=1cm]{geometry}
\pagestyle{empty}

\usepackage{setspace}
\setlength\parskip{0pt}
\setlength\parindent{0pt}

\usepackage{graphicx, xcolor, wallpaper}
\usepackage{tikz}
\usetikzlibrary{arrows, decorations.markings, positioning}
\definecolor{cevaA}{HTML}{8B6639}
\definecolor{cevaB}{HTML}{029044}
\definecolor{daf}{HTML}{C0DA91}

%\pagecolor{cevaA}
\pagecolor{daf}
\color{cevaA}

%\usepackage{eso-pic}
%\newcommand\BackgroundPic{
%\put(0,0){
%\parbox[b][\paperheight]{\paperwidth}{%
%\vfill
%\centering
%\includegraphics[width=\paperwidth,height=\paperheight]{nijar.jpg}%
%\vfill
%}}}

\usepackage{fontspec}
\usepackage[RTLdocument]{bidi}
\setmainfont[Script=Hebrew,
	AutoFakeSlant=-0.15,
	BoldFont={Rutz_OE Bold Pro}]
	{Rutz_OE Regular Pro}
	\setmonofont[Scale=0.8]{PragmataPro}
\newcommand{\hl}[1]{\textbf{\addfontfeature{Color=cevaB}#1}}
\renewcommand{\L}[1]{\LR{\fontspec{Vesper Pro}#1}}
\newcommand{\entypo}[1]{{\fontspec{Entypo}#1}}
\newcommand{\symbolglyph}[1]{{\fontspec[Color=cevaB]{Symbola}#1}}
\usepackage{url}

% for double arrows a la chef
% adapt line thickness and line width, if needed
%\tikzstyle{vecArrow} = [thick, decoration={markings,mark=at position 1 with {\arrow[semithick]{open triangle 60}}},
%   double distance=1.4pt, shorten >= 5.5pt,
%   preaction = {decorate},
%   postaction = {draw,line width=1.4pt, white,shorten >= 4.5pt}]
\tikzstyle{vecArrow} = [
	line width = 8pt,
	decoration={markings, mark = at position 1 with {\arrow[line width=5pt, color = cevaA!50!cevaB]{triangle 60}}},
	shorten >= 5.5pt,
	color = cevaA!50!cevaB,
	preaction = {decorate}
]



\begin{document}
\TileWallPaper{20pt}{20pt}{nekuda.eps} % By Mauricio Duque <http://www.snap2objects.com/2009/05/free-clothing-vector-silhouettes/>
%\AddToShipoutPicture{\BackgroundPic}

\begin{center}
\color{cevaA}
\fontsize{30}{36}\selectfont

\begin{tikzpicture}[thick]
	\node (a) {\parbox{15cm}{\centering\setRL\color{cevaA} בחינוך ביתי ומחפשים \hl{חֶבְרָה}?}};
	\node [below = 0.0cm of a] (b) {\parbox{15cm}{\centering\setRL\color{cevaA} רוצים להפגש עם משפחות\\
	נוספות עם ילדים \hl{בשעות הבוקר}?}};
	\node [inner sep=0,minimum size=0,right = of a] (x) {}; % invisible node
	\node [inner sep=0,minimum size=0,left = of b] (y) {}; % invisible node
	%\draw [vecArrow] (a) to (x);
	\draw [vecArrow] (a) |- (x) |- (b);
	\draw [vecArrow] (b) to (y) |- (a);
	%\draw [->] (y) |- (a);
\end{tikzpicture}

\vfill

\vspace{-1cm}
{\fontsize{50}{60}\selectfont
\textbf{\textcolor{cevaA}{אתם מוזמנים להצטרף}}\\
{\fontspec[Script=Hebrew]{HPLJPM+AlenbiSerif}\fontsize{75}{90}\selectfont \textcolor{cevaA}{ל}\textcolor{cevaB}{מפגשים קבועים}}\\
\vspace{-0.2em}\textbf{ורב־גילאיים \hl{בבקרים} של ימי}\\
\vspace{0.1em}\textbf{ראשון ורביעי \hl{באיזור}}
}

\vfill

כל הפרטים, בדף שלנו ב\hl{אינטרנט}:


\includegraphics[width=0.20\textwidth]{qr-bkarim.png}\\
\vspace{-0.5cm}\url{tiny.cc/bkarim}

\vfill

לפרטים ב\hl{טלפון} או ב\hl{דואל}\\אפשר לפנות אל יודה רונן:

\vspace{0.5cm}
\begin{LTR}
\begin{tabular}{cc}
	\symbolglyph{☏} & 02-6419913\\
	\symbolglyph{📱} & 054-6509361\\
	\symbolglyph{@} & {\url{foo@digitalwords.net}}
\end{tabular}
\end{LTR}

\end{center}
\end{document}
